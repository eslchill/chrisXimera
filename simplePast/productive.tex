\documentclass{ximera}
%% handout
%% nohints
%% space
%% newpage
%% numbers

%%%% You can put user macros here
%% However, you cannot make new environments

\graphicspath{{./}{firstExample/}{secondExample/}}

\usepackage{tikz}
\usepackage{tkz-euclide}
\usetkzobj{all}

\tikzstyle geometryDiagrams=[ultra thick,color=blue!50!black]
 %% we can turn off input when making a master document

\outcome{Produce simple past tense verbs.}

\title{Simple Past - Productive}
\begin{document}
\begin{abstract}
Make some simple past verbs.
\end{abstract}
\maketitle

\textbf{Complete the paragraph with past tense forms of the verbs in parentheses.}

\begin{question}
\textbf{All answers in one question, no solution:}

Yesterday, Jim \answer{went} (go) to the store. He \answer{needed} (need) to buy some groceries. There \answer{was} (be) a big sale on cheese, so he \answer{bought} (buy) several wheels. His reusable grocery bags \answer{were} (be) very heavy but he \answer{managed} (manage) to carry them home. He \answer{ate} (eat) one wheel and \answer{put} (put) the others into his fridge. He \answer{survived} (survive) this experience, but barely.

\end{question}

\rule{1cm}{1pt}

\begin{question}
\begin{solution}

\textbf{All answers in one question, one solution:}

Yesterday, Jim \answer{went} (go) to the store. He \answer{needed} (need) to buy some groceries. There \answer{was} (be) a big sale on cheese, so he \answer{bought} (buy) several wheels. His reusable grocery bags \answer{were} (be) very heavy but he \answer{managed} (manage) to carry them home. He \answer{ate} (eat) one wheel and \answer{put} (put) the others into his fridge. He \answer{survived} (survive) this experience, but barely.

\end{solution}
\end{question}

\rule{1cm}{1pt}

\begin{question}
\begin{solution}

\textbf{All answers in one question, many solutions:}

Yesterday, Jim \answer{went} (go) to the store.\end{solution}\begin{solution} He \answer{needed} (need) to buy some groceries.\end{solution}\begin{solution} There \answer{was} (be) a big sale on cheese, so he \answer{bought} (buy) several wheels.\end{solution}\begin{solution} His reusable grocery bags \answer{were} (be) very heavy but he \answer{managed} (manage) to carry them home.\end{solution}\begin{solution} He \answer{ate} (eat) one wheel\end{solution}\begin{solution} and \answer{put} (put) the others into his fridge.\end{solution}\begin{solution} He \answer{survived} (survive) this experience, but barely.

\end{solution}
\end{question}

\rule{1cm}{1pt}

\begin{question}

\textbf{All answers in many questions, one answer each, solutions omitted:}

Yesterday, Jim \answer{went} (go) to the store.\end{question}\begin{question} He \answer{needed} (need) to buy some groceries.\end{question}\begin{question} There \answer{was} (be) a big sale on cheese, so he \answer{bought} (buy) several wheels.\end{question}\begin{question} His reusable grocery bags \answer{were} (be) very heavy but he \answer{managed} (manage) to carry them home.\end{question}\begin{question} He \answer{ate} (eat) one wheel\end{question}\begin{question} and \answer{put} (put) the others into his fridge.\end{question} \begin{question} He \answer{survived} (survive) this experience, but barely.

\end{question}

\rule{1cm}{1pt}

\begin{question}
\begin{solution}
\textbf{Yesterday, Bart went to the store.}
What tense is the verb in this sentence?\answer{past}
\begin{hint}
This action is complete.
\end{hint}
\begin{hint}
This action occurred yesterday.
\end{hint}
\begin{hint}
Yesterday is in the past.
\end{hint}
\end{solution}
\end{question}

\end{document}